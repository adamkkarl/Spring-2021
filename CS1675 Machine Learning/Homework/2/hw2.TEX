\documentclass[a4paper]{article}

%% Language and font encodings
\usepackage[english]{babel}
\usepackage[utf8x]{inputenc}
\usepackage[T1]{fontenc}
\usepackage{graphicx} 

%% Sets page size and margins
\usepackage[a4paper,top=3cm,bottom=2cm,left=3cm,right=3cm,marginparwidth=1.75cm]{geometry}

%% Useful packages
\usepackage{amsmath}
\usepackage{graphicx}
\usepackage[colorinlistoftodos]{todonotes}
\usepackage[colorlinks=true, allcolors=blue]{hyperref}

\title{Homework 2}
\author{Adam Karl}

\begin{document}
\maketitle

\section{Problem 1}
Part 1

\begin{itemize}
    \item mean = 15.0415
    \item standard deviation = 5.0279
\end{itemize}
The calculated mean and standard deviations are each a few hundredths off of the true values. This is likely because matlab is rounding our input values to a certain number of decimal places.

\medskip
\noindent
Part 4


\begin{center}
    \includegraphics[scale=1]{meansHistogram25.png}
    \caption{Subsample size = 25}
\end{center}
The means have a roughly normal distribution around the true mean of 15.

\noindent
The mean of all subsample means was calculated to be 15.0587.

  
\bigskip
\noindent
Part 5

\begin{center}
    \includegraphics[scale=1]{meansHistogram40.png}
    \caption{Subsample size = 40}
\end{center}

\noindent 
The means for a subsample size of 40 have a marginally tighter grouping than the 25-size subsample means around the true mean of 15. It's possible I was unlucky with my samples, but a subsample size of 40 does not appear to be much more accurate than a subsample size of 25.

\noindent
The mean of all subsample means was calculated to be 15.0371.

\bigskip
\noindent
Part 6

\noindent
The first 25 elements have a mean of 14.5625.

\smallskip
\noindent
The 95\% confidence interval has a range of 12.6392 to 16.4974. The true mean of 15 is well within these bounds.

\section{Problem 2}

\begin{itemize}
    \item m=1
    \begin{itemize}
        \item test mean = 2.1346
        \item test std = 3.2383
    \end{itemize}
    \item m=2
    \begin{itemize}
        \item test mean = 2.1077
        \item test std = 3.2509
    \end{itemize}
    \item m=3
    \begin{itemize}
        \item test mean = 2.1798
        \item test std = 3.2644
    \end{itemize}
    \item m=4
    \begin{itemize}
        \item test mean = 1.9794
        \item test std = 3.2421
    \end{itemize}
    \item m=5
    \begin{itemize}
        \item test mean = 2.0685
        \item test std = 3.2890
    \end{itemize}
    \item m=6
    \begin{itemize}
        \item test mean = 2.0369
        \item test std = 3.3452
    \end{itemize}
    \item m=7
    \begin{itemize}
        \item test mean = 2.1467
        \item test std = 3.3005
    \end{itemize}
    \item m=8
    \begin{itemize}
        \item test mean = 2.0701
        \item test std = 3.3043
    \end{itemize}
    \item m=9
    \begin{itemize}
        \item test mean = 1.9808
        \item test std = 3.2341
    \end{itemize}
    \item m=10
    \begin{itemize}
        \item test mean = 2.0516
        \item test std = 3.1796
    \end{itemize}
\end{itemize}

\section{Problem 3}
a.
\begin{itemize}
    \item p(2) = 1/36
    \begin{itemize}
        \item 1 way: (1,1)
    \end{itemize}
    \item p(3) = 2/36 = 1/18
    \begin{itemize}
        \item 2 ways: (1,2), (2,1)
    \end{itemize}
    \item p(4) = 3/36 = 1/12
    \begin{itemize}
        \item 3 ways: (1,3), (2,2), (3,1)
    \end{itemize}    
    \item p(5) = 4/36 = 1/9
    \begin{itemize}
        \item 4 ways: (1,4), (2,3), (3,2), (4,1)
    \end{itemize}   
    \item p(6) = 5/36
    \begin{itemize}
        \item 5 ways: (1,5), (2,4), (3,3), (4,2), (5,1)
    \end{itemize} 
    \item p(7) = 6/36 = 1/6
    \begin{itemize}
        \item 6 ways: (1,6), (2,5), (3,4), (4,3), (5,2), (6,1)
    \end{itemize} 
    \item p(8) = 5/36
    \begin{itemize}
        \item 5 ways: (2,6), (3,5), (4,4), (5,3), (6,2)
    \end{itemize} 
    \item p(9) = 4/36 = 1/9
    \begin{itemize}
        \item 4 ways: (3,6), (4,5), (5,4), (6,3)
    \end{itemize}   
    \item p(10) = 3/36 = 1/12
    \begin{itemize}
        \item 3 ways: (4,6), (5,5), (6,4)
    \end{itemize} 
    \item p(11) = 2/36 = 1/18
    \begin{itemize}
        \item 2 ways: (5,6), (6,5)
    \end{itemize}
    \item p(12) = 1/36
    \begin{itemize}
        \item 1 way: (6,6)
    \end{itemize}
\end{itemize}

\noindent
b. EV $= 2(1/36) + 3(1/18) + 4(1/12) + 5(1/9) + 6(5/36) + 7(1/6) + 8(5/36) + 9(1/9) + 10(1/12) + 11(1/18) + 12(1/36) = 7$

\medskip
\noindent
c. The outcome of 4 has a 1/12 chance. Therefore, the probability of NOT 4 is $1-1/12=11/12$. The probability of NOT 4, five consecutive times is $(11/12)^5 \approx 0.64722$.

\medskip
\noindent
The probability of an odd outcome is equal to p(3) + p(5) + p(7) + p(9) + p(11) = $1/18+1/9+1/6+1/9+1/18=0.5$. The chance of an odd outcome all 5 trials is $0.5^5=0.03125$ (or 1/32).

\section{Problem 4}
Applying Bayes' theorem:

\[P(infected|positive) = \frac{P(positive| infected)P(infected)}{P(positive)}\]

\[P(infected|positive) = \frac{P(positive| infected)P(infected)}{P(false positive) + P(true positive)}\]

\[P(infected|positive) = \frac{0.99*0.0001}{0.9999*0.01 + 0.0001*0.99}\]

\[P(infected|positive) = \frac{0.99*0.0001}{0.9999*0.01 + 0.0001*0.99}\]

\[P(infected|positive) \approx 0.0098\]

\noindent
Since the probability that a person that tested positive is actually infected is only about 1\%, I cannot recommend this test be widely adopted.

\section{Problem 5}
a. Simply integrate the probability density function over the range (a to b):

\[ \int_{a}^{b} \frac{1}{b-a} \,dx \]

\[ \frac{1}{b-a} \int_{a}^{b} 1  \,dx \]

\[ (\frac{1}{b-a}) x \Big|_a^b  \]

\[ (\frac{1}{b-a}) (b-a)  \]

\[ 1 \]

\medskip
\noindent
b. For a continuous distribution, to get the mean we integrate $x*f(x)$ over the entire range (here, from a to b):

\[ \int_{a}^{b} x\frac{1}{b-a} \,dx \]

\[ \frac{1}{b-a} \int_{a}^{b} x \,dx \]

\[ (\frac{1}{b-a}) \frac{1}{2}x^2 \Big|_a^b  \]

\[ (\frac{1}{b-a}) \frac{1}{2}(b^2-a^2)  \]

\[ (\frac{1}{b-a}) \frac{1}{2}(b+a)(b-a)  \]

\[ (\frac{1}{b-a}) \frac{1}{2}(b+a)(b-a)  \]

\[ \frac{b+a}{2}  \]

\noindent
This makes logical sense, since it is the midpoint of the uniform distribution.

\section{Problem 6}

a. 65/100 = 0.65

\noindent
b. 

\begin{center}
    \includegraphics[scale=1]{6-b-prior.png}
    \caption{prior}
\end{center}

The prior distribution an a uniform distribution from 0 to 1.

\begin{center}
    \includegraphics[scale=1]{6-b.png}
    \caption{posterior}
\end{center}

The posterior distribution is a normal-like distribution centered on our estimated theta of 0.65.

\noindent
c. 
\[ \theta_{MAP} = \frac{N_1+\alpha_1-1}{N_1+N_2+\alpha_1+\alpha_2-2} \]

\[ \theta_{MAP} =\frac{65+1-1}{65+35+1+1-2} \]

\[ \theta_{MAP} =\frac{65}{100}=0.65 \]

\begin{center}
    \includegraphics[scale=1]{6-c.png}
    \caption{$\theta_{MAP} = 0.65$}
\end{center}



\noindent 
d. 

\begin{center}
    \includegraphics[scale=1]{6-d-prior.png}
    \caption{prior}
\end{center}

\[ \theta_{MAP} = \frac{N_1+\alpha_1-1}{N_1+N_2+\alpha_1+\alpha_2-2} \]

\[ \theta_{MAP} =\frac{65+4-1}{65+35+4+2-2} \]

\[ \theta_{MAP} =\frac{68}{104}\approx 0.6538 \]

\begin{center}
    \includegraphics[scale=1]{6-d-posterior.png}
    \caption{$\theta_{map}\approx 0.6538$} 
\end{center}

\end{document}
