\documentclass{article}

%% Language and font encodings
\usepackage[english]{babel}
\usepackage[utf8x]{inputenc}
\usepackage[T1]{fontenc}
\usepackage{graphicx} 

%% Sets page size and margins
\usepackage[a4paper,top=3cm,bottom=2cm,left=3cm,right=3cm,marginparwidth=1.75cm]{geometry}

%% Useful packages
\usepackage{amsmath}
\usepackage{graphicx}
\usepackage[colorinlistoftodos]{todonotes}
\usepackage[colorlinks=true, allcolors=blue]{hyperref}

\title{HW 8}
\author{Adam Karl}

\begin{document}

\maketitle

\section{Bayesian Belief Networks}
a. $P(B=T,E=T)=$
\[\sum_{a\epsilon T,F}\sum_{d\epsilon T,F,X}\sum_{c\epsilon T,F}\sum_{f\epsilon T,F} P(A=a) P(B=T) P(C=c) P(D=d|A=a,B=T,C=c) P(E=T|C=c) P(F=f|D=d) \]

\noindent
There are 4 binary variables and 1 tertiary variable, so the number of parameters for the full joint is $2^4(3^1)=48$. There are 23 addends and 120 multiplicands for this calculation.

\noindent 
b. We can group together the variables that depend on each other.

\[\sum_{f\epsilon T,F}P(F=f|D=d)\sum_{d\epsilon T,F,X} P(D=d|A=a,B=T,C=c) P(B=T) \]

\[[\sum_{a\epsilon T,F}P(A=a)] P(E=T|C=c)[\sum_{c\epsilon T,F}P(C=c)]\]

\noindent 
This leaves us with 25 addends and 8 multiplicands, which is more efficient for addition but less efficient for multiplication.

\section{Pneumonia Diagnosis}

a.
\begin{itemize}
    \item pneumonia = unknown
    \begin{itemize}
        \item P(fever) = 0.606
        \item P(paleness) = 0.504
        \item P(cough) = 0.116
        \item P(highWBcount) = 0.506
        \item P(pneumonia) = 0.02
    \end{itemize}
    \item pneumonia = true
    \begin{itemize}
        \item P(fever) = 0.9
        \item P(paleness) = 0.7
        \item P(cough) = 0.9
        \item P(highWBcount) = 0.8
    \end{itemize}
    \item pneumonia = false
    \begin{itemize}
        \item P(fever) = 0.6
        \item P(paleness) = 0.5
        \item P(cough) = 0.1
        \item P(highWBcount) = 0.5
    \end{itemize}
\end{itemize}

\noindent
b. 
\[\frac{P(fever=T|pneu=T)P(paleness=F|pneu=T)P(cough=T|pneu=T)P(highWBcount=F|pneu=T)P(pneu=t) }{P(fever=T)P(paleness=F)P(cough=T)P(highWBcount=F)}\]

\[\frac{0.9(0.3)(0.9)(0.2)(0.02) }{0.606(0.496)(0.116)(0.494)} = 0.0564\]

\noindent
Therefore, there is a 5.64\% chance the patient has pneumonia.

\noindent 
c. 
\[\frac{P(fever=T|pneu=T)P(cough=T|pneu=T)P(pneu=t) }{P(fever=T)P(cough=T)}\]
\[\frac{0.9(0.9)(0.02) }{(0.606)(0.116)} = 0.2304\]

\noindent
Therefore, there is a 23.04\% chance the patient has pneumonia.


\end{document}

